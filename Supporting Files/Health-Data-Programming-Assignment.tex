% Options for packages loaded elsewhere
\PassOptionsToPackage{unicode}{hyperref}
\PassOptionsToPackage{hyphens}{url}
%
\documentclass[
]{article}
\title{Observations of Injury Types Amongst NHS Patients Residing in
Scotland (2011-2022)}
\author{Exam Number: B215161}
\date{23/06/2023}

\usepackage{amsmath,amssymb}
\usepackage{lmodern}
\usepackage{iftex}
\ifPDFTeX
  \usepackage[T1]{fontenc}
  \usepackage[utf8]{inputenc}
  \usepackage{textcomp} % provide euro and other symbols
\else % if luatex or xetex
  \usepackage{unicode-math}
  \defaultfontfeatures{Scale=MatchLowercase}
  \defaultfontfeatures[\rmfamily]{Ligatures=TeX,Scale=1}
\fi
% Use upquote if available, for straight quotes in verbatim environments
\IfFileExists{upquote.sty}{\usepackage{upquote}}{}
\IfFileExists{microtype.sty}{% use microtype if available
  \usepackage[]{microtype}
  \UseMicrotypeSet[protrusion]{basicmath} % disable protrusion for tt fonts
}{}
\makeatletter
\@ifundefined{KOMAClassName}{% if non-KOMA class
  \IfFileExists{parskip.sty}{%
    \usepackage{parskip}
  }{% else
    \setlength{\parindent}{0pt}
    \setlength{\parskip}{6pt plus 2pt minus 1pt}}
}{% if KOMA class
  \KOMAoptions{parskip=half}}
\makeatother
\usepackage{xcolor}
\IfFileExists{xurl.sty}{\usepackage{xurl}}{} % add URL line breaks if available
\IfFileExists{bookmark.sty}{\usepackage{bookmark}}{\usepackage{hyperref}}
\hypersetup{
  pdftitle={Observations of Injury Types Amongst NHS Patients Residing in Scotland (2011-2022)},
  pdfauthor={Exam Number: B215161},
  hidelinks,
  pdfcreator={LaTeX via pandoc}}
\urlstyle{same} % disable monospaced font for URLs
\usepackage[margin=1in]{geometry}
\usepackage{color}
\usepackage{fancyvrb}
\newcommand{\VerbBar}{|}
\newcommand{\VERB}{\Verb[commandchars=\\\{\}]}
\DefineVerbatimEnvironment{Highlighting}{Verbatim}{commandchars=\\\{\}}
% Add ',fontsize=\small' for more characters per line
\usepackage{framed}
\definecolor{shadecolor}{RGB}{248,248,248}
\newenvironment{Shaded}{\begin{snugshade}}{\end{snugshade}}
\newcommand{\AlertTok}[1]{\textcolor[rgb]{0.94,0.16,0.16}{#1}}
\newcommand{\AnnotationTok}[1]{\textcolor[rgb]{0.56,0.35,0.01}{\textbf{\textit{#1}}}}
\newcommand{\AttributeTok}[1]{\textcolor[rgb]{0.77,0.63,0.00}{#1}}
\newcommand{\BaseNTok}[1]{\textcolor[rgb]{0.00,0.00,0.81}{#1}}
\newcommand{\BuiltInTok}[1]{#1}
\newcommand{\CharTok}[1]{\textcolor[rgb]{0.31,0.60,0.02}{#1}}
\newcommand{\CommentTok}[1]{\textcolor[rgb]{0.56,0.35,0.01}{\textit{#1}}}
\newcommand{\CommentVarTok}[1]{\textcolor[rgb]{0.56,0.35,0.01}{\textbf{\textit{#1}}}}
\newcommand{\ConstantTok}[1]{\textcolor[rgb]{0.00,0.00,0.00}{#1}}
\newcommand{\ControlFlowTok}[1]{\textcolor[rgb]{0.13,0.29,0.53}{\textbf{#1}}}
\newcommand{\DataTypeTok}[1]{\textcolor[rgb]{0.13,0.29,0.53}{#1}}
\newcommand{\DecValTok}[1]{\textcolor[rgb]{0.00,0.00,0.81}{#1}}
\newcommand{\DocumentationTok}[1]{\textcolor[rgb]{0.56,0.35,0.01}{\textbf{\textit{#1}}}}
\newcommand{\ErrorTok}[1]{\textcolor[rgb]{0.64,0.00,0.00}{\textbf{#1}}}
\newcommand{\ExtensionTok}[1]{#1}
\newcommand{\FloatTok}[1]{\textcolor[rgb]{0.00,0.00,0.81}{#1}}
\newcommand{\FunctionTok}[1]{\textcolor[rgb]{0.00,0.00,0.00}{#1}}
\newcommand{\ImportTok}[1]{#1}
\newcommand{\InformationTok}[1]{\textcolor[rgb]{0.56,0.35,0.01}{\textbf{\textit{#1}}}}
\newcommand{\KeywordTok}[1]{\textcolor[rgb]{0.13,0.29,0.53}{\textbf{#1}}}
\newcommand{\NormalTok}[1]{#1}
\newcommand{\OperatorTok}[1]{\textcolor[rgb]{0.81,0.36,0.00}{\textbf{#1}}}
\newcommand{\OtherTok}[1]{\textcolor[rgb]{0.56,0.35,0.01}{#1}}
\newcommand{\PreprocessorTok}[1]{\textcolor[rgb]{0.56,0.35,0.01}{\textit{#1}}}
\newcommand{\RegionMarkerTok}[1]{#1}
\newcommand{\SpecialCharTok}[1]{\textcolor[rgb]{0.00,0.00,0.00}{#1}}
\newcommand{\SpecialStringTok}[1]{\textcolor[rgb]{0.31,0.60,0.02}{#1}}
\newcommand{\StringTok}[1]{\textcolor[rgb]{0.31,0.60,0.02}{#1}}
\newcommand{\VariableTok}[1]{\textcolor[rgb]{0.00,0.00,0.00}{#1}}
\newcommand{\VerbatimStringTok}[1]{\textcolor[rgb]{0.31,0.60,0.02}{#1}}
\newcommand{\WarningTok}[1]{\textcolor[rgb]{0.56,0.35,0.01}{\textbf{\textit{#1}}}}
\usepackage{graphicx}
\makeatletter
\def\maxwidth{\ifdim\Gin@nat@width>\linewidth\linewidth\else\Gin@nat@width\fi}
\def\maxheight{\ifdim\Gin@nat@height>\textheight\textheight\else\Gin@nat@height\fi}
\makeatother
% Scale images if necessary, so that they will not overflow the page
% margins by default, and it is still possible to overwrite the defaults
% using explicit options in \includegraphics[width, height, ...]{}
\setkeys{Gin}{width=\maxwidth,height=\maxheight,keepaspectratio}
% Set default figure placement to htbp
\makeatletter
\def\fps@figure{htbp}
\makeatother
\setlength{\emergencystretch}{3em} % prevent overfull lines
\providecommand{\tightlist}{%
  \setlength{\itemsep}{0pt}\setlength{\parskip}{0pt}}
\setcounter{secnumdepth}{-\maxdimen} % remove section numbering
\usepackage{booktabs}
\usepackage{longtable}
\usepackage{array}
\usepackage{multirow}
\usepackage{wrapfig}
\usepackage{float}
\usepackage{colortbl}
\usepackage{pdflscape}
\usepackage{tabu}
\usepackage{threeparttable}
\usepackage{threeparttablex}
\usepackage[normalem]{ulem}
\usepackage{makecell}
\usepackage{xcolor}
\ifLuaTeX
  \usepackage{selnolig}  % disable illegal ligatures
\fi

\begin{document}
\maketitle

{
\setcounter{tocdepth}{3}
\tableofcontents
}
\#\#Title: Observations of Injury Types Amongst NHS Patients Residing in
Scotland (2011-2022)

\#\#Overview:

The data presented is a joined data set using three csv files provided
by different NHS Services across Scotland. This data has been collected
over the years of 2011 to 2022 and includes demographics defined by sex
(male and female) and age groups (0-4 years, 5-9 years, 10-14 years,
15-24 years, 25-44 years, 45-64 years, 65-74 years, 75+ years))

\#Most Common Injury Observed amongst Age and Sex:

It can be observed from the data sets, which span from 2011-2022 the
most frequent injury suffered is has been recorded as a falling injury,
the demographic who has been particularly affected by this type of
accident is specifically females aged 75+. Followed by male and females
aged between 45 to 65 years. The graph has been plotted using a
logarithmic scale due to the high volume of admission data collected
concerning falling injury. It is clear to see that this type of injury
is the most prevalent amongst the Scottish NHS services.

\#Rate of Death by Admission for Falling Injuries:

It can be observed that there is a positive correlation between the rate
of death by quantity of admissions. This does not mean however the
quantity of deaths is the same as the quantity of admissions, but that
there is more deaths recorded through a greater number of recorded
admissions across the NHS services of Scotland between the years
2011-2022.

\#\#Data Processing:

\begin{Shaded}
\begin{Highlighting}[]
\FunctionTok{library}\NormalTok{(tidyverse)}
\end{Highlighting}
\end{Shaded}

\begin{verbatim}
## -- Attaching packages --------------------------------------- tidyverse 1.3.2 --
## v ggplot2 3.3.6     v purrr   0.3.4
## v tibble  3.1.7     v dplyr   1.0.9
## v tidyr   1.2.0     v stringr 1.4.0
## v readr   2.1.2     v forcats 0.5.1
## -- Conflicts ------------------------------------------ tidyverse_conflicts() --
## x dplyr::filter() masks stats::filter()
## x dplyr::lag()    masks stats::lag()
\end{verbatim}

\begin{Shaded}
\begin{Highlighting}[]
\FunctionTok{library}\NormalTok{(lubridate)}
\end{Highlighting}
\end{Shaded}

\begin{verbatim}
## 
## Attaching package: 'lubridate'
## 
## The following objects are masked from 'package:base':
## 
##     date, intersect, setdiff, union
\end{verbatim}

\begin{Shaded}
\begin{Highlighting}[]
\FunctionTok{library}\NormalTok{(magrittr)}
\end{Highlighting}
\end{Shaded}

\begin{verbatim}
## 
## Attaching package: 'magrittr'
## 
## The following object is masked from 'package:purrr':
## 
##     set_names
## 
## The following object is masked from 'package:tidyr':
## 
##     extract
\end{verbatim}

\begin{Shaded}
\begin{Highlighting}[]
\FunctionTok{library}\NormalTok{(dplyr)}
\FunctionTok{library}\NormalTok{(ggplot2)}
\FunctionTok{library}\NormalTok{(janitor)}
\end{Highlighting}
\end{Shaded}

\begin{verbatim}
## 
## Attaching package: 'janitor'
## 
## The following objects are masked from 'package:stats':
## 
##     chisq.test, fisher.test
\end{verbatim}

\begin{Shaded}
\begin{Highlighting}[]
\FunctionTok{library}\NormalTok{(kableExtra)}
\end{Highlighting}
\end{Shaded}

\begin{verbatim}
## 
## Attaching package: 'kableExtra'
## 
## The following object is masked from 'package:dplyr':
## 
##     group_rows
\end{verbatim}

\begin{Shaded}
\begin{Highlighting}[]
\FunctionTok{library}\NormalTok{(scales)}
\end{Highlighting}
\end{Shaded}

\begin{verbatim}
## 
## Attaching package: 'scales'
## 
## The following object is masked from 'package:purrr':
## 
##     discard
## 
## The following object is masked from 'package:readr':
## 
##     col_factor
\end{verbatim}

\begin{Shaded}
\begin{Highlighting}[]
\FunctionTok{library}\NormalTok{(viridis)}
\end{Highlighting}
\end{Shaded}

\begin{verbatim}
## Loading required package: viridisLite
## 
## Attaching package: 'viridis'
## 
## The following object is masked from 'package:scales':
## 
##     viridis_pal
\end{verbatim}

\begin{Shaded}
\begin{Highlighting}[]
\CommentTok{\#Calling necessary packages for RMarkdown.}

\CommentTok{\#getwd() if necessary}
\CommentTok{\#setwd() if necessary}

\NormalTok{orig\_hb }\OtherTok{=} \FunctionTok{read.csv}\NormalTok{(}\StringTok{\textquotesingle{}hb\_lookup.csv\textquotesingle{}}\NormalTok{)}
\NormalTok{orig\_uiadmissions }\OtherTok{=} \FunctionTok{read.csv}\NormalTok{(}\StringTok{\textquotesingle{}ui\_admissions\_2022.csv\textquotesingle{}}\NormalTok{)}
\NormalTok{orig\_uideaths }\OtherTok{=} \FunctionTok{read.csv}\NormalTok{(}\StringTok{\textquotesingle{}ui\_deaths\_2022.csv\textquotesingle{}}\NormalTok{)}

\CommentTok{\#CSV files read into RMarkdown and assigned to object names for ease of use.}

\FunctionTok{glimpse}\NormalTok{(orig\_hb) }
\end{Highlighting}
\end{Shaded}

\begin{verbatim}
## Rows: 18
## Columns: 5
## $ HB             <chr> "S08000015", "S08000016", "S08000017", "S08000018", "S0~
## $ HBName         <chr> "NHS Ayrshire and Arran", "NHS Borders", "NHS Dumfries ~
## $ HBDateEnacted  <int> 20140401, 20140401, 20140401, 20140401, 20140401, 20140~
## $ HBDateArchived <int> NA, NA, NA, 20180201, NA, NA, 20190331, NA, 20190331, N~
## $ Country        <chr> "S92000003", "S92000003", "S92000003", "S92000003", "S9~
\end{verbatim}

\begin{Shaded}
\begin{Highlighting}[]
\CommentTok{\#Rows: 18 Columns: 5, HB numbers + name(location in Scotland). Couple of "NA"s in column 4, consistent result in 3}

\FunctionTok{glimpse}\NormalTok{(orig\_uiadmissions) }
\end{Highlighting}
\end{Shaded}

\begin{verbatim}
## Rows: 391,104
## Columns: 14
## $ FinancialYear      <chr> "2011/12", "2011/12", "2011/12", "2011/12", "2011/1~
## $ HBR                <chr> "S92000003", "S92000003", "S92000003", "S92000003",~
## $ HBRQF              <chr> "d", "d", "d", "d", "d", "d", "d", "d", "d", "d", "~
## $ CA                 <chr> "S92000003", "S92000003", "S92000003", "S92000003",~
## $ CAQF               <chr> "d", "d", "d", "d", "d", "d", "d", "d", "d", "d", "~
## $ AgeGroup           <chr> "All", "All", "All", "All", "All", "All", "All", "A~
## $ AgeGroupQF         <chr> "d", "d", "d", "d", "d", "d", "d", "d", "d", "d", "~
## $ Sex                <chr> "All", "All", "All", "All", "All", "All", "All", "A~
## $ SexQF              <chr> "d", "d", "d", "d", "d", "d", "d", "d", "d", "d", "~
## $ InjuryLocation     <chr> "All", "All", "All", "All", "All", "All", "All", "A~
## $ InjuryLocationQF   <chr> "d", "d", "d", "d", "d", "d", "d", "d", "d", "", ""~
## $ InjuryType         <chr> "All Diagnoses", "RTA", "Poisoning", "Falls", "Stru~
## $ InjuryTypeQF       <chr> "d", "", "", "", "", "", "", "", "", "d", "", "", "~
## $ NumberOfAdmissions <int> 54766, 3058, 1989, 32694, 2738, 1061, 481, 5504, 73~
\end{verbatim}

\begin{Shaded}
\begin{Highlighting}[]
\CommentTok{\#Rows: 391,104 Columns: 14, Breakdown of HB, Year, Age, Sex, Injurylocation, Injurytype, Number of Admissions | Data is raw and aggregated. Columns with repeated results (c(3,5,7,9,11,13)), all stating "QF". Summarised values in columns (c(6,8,10,12)), stating "All".}

\FunctionTok{glimpse}\NormalTok{(orig\_uideaths)}
\end{Highlighting}
\end{Shaded}

\begin{verbatim}
## Rows: 182,502
## Columns: 14
## $ Year             <int> 2011, 2011, 2011, 2011, 2011, 2011, 2011, 2011, 2011,~
## $ HBR              <chr> "S92000003", "S92000003", "S92000003", "S92000003", "~
## $ HBRQF            <chr> "d", "d", "d", "d", "d", "d", "d", "d", "d", "d", "d"~
## $ CA               <chr> "S92000003", "S92000003", "S92000003", "S92000003", "~
## $ CAQF             <chr> "d", "d", "d", "d", "d", "d", "d", "d", "d", "d", "d"~
## $ AgeGroup         <chr> "All", "All", "All", "All", "All", "All", "All", "All~
## $ AgeGroupQF       <chr> "d", "d", "d", "d", "d", "d", "d", "d", "d", "d", "d"~
## $ Sex              <chr> "All", "All", "All", "All", "All", "All", "All", "All~
## $ SexQF            <chr> "d", "d", "d", "d", "d", "d", "d", "d", "d", "d", "d"~
## $ InjuryLocation   <chr> "All", "All", "All", "All", "All", "All", "All", "All~
## $ InjuryLocationQF <chr> "d", "d", "d", "d", "d", "d", "d", "d", "d", "", "", ~
## $ InjuryType       <chr> "All", "Land transport accidents", "Poisoning", "Fall~
## $ InjuryTypeQF     <chr> "d", "", "", "", "", "", "", "", "", "d", "", "", "",~
## $ NumberofDeaths   <int> 1657, 204, 461, 723, 6, 1, 2, 72, 188, 569, 0, 366, 1~
\end{verbatim}

\begin{Shaded}
\begin{Highlighting}[]
\CommentTok{\#Rows: 182,502 Columns: 14, Breakdown of HB, Year, Age, Sex, Injurylocation, Injurytype, Number of Deaths | Data is raw and aggregated}

\FunctionTok{summary}\NormalTok{(orig\_hb)}
\end{Highlighting}
\end{Shaded}

\begin{verbatim}
##       HB               HBName          HBDateEnacted      HBDateArchived    
##  Length:18          Length:18          Min.   :20140401   Min.   :20180201  
##  Class :character   Class :character   1st Qu.:20140401   1st Qu.:20180201  
##  Mode  :character   Mode  :character   Median :20140401   Median :20185266  
##                                        Mean   :20150379   Mean   :20185266  
##                                        3rd Qu.:20140401   3rd Qu.:20190331  
##                                        Max.   :20190401   Max.   :20190331  
##                                                           NA's   :14        
##    Country         
##  Length:18         
##  Class :character  
##  Mode  :character  
##                    
##                    
##                    
## 
\end{verbatim}

\begin{Shaded}
\begin{Highlighting}[]
\CommentTok{\#HB and HBName will be required for joining datasets after cleaning.}

\FunctionTok{summary}\NormalTok{(orig\_uiadmissions)}
\end{Highlighting}
\end{Shaded}

\begin{verbatim}
##  FinancialYear          HBR               HBRQF                CA           
##  Length:391104      Length:391104      Length:391104      Length:391104     
##  Class :character   Class :character   Class :character   Class :character  
##  Mode  :character   Mode  :character   Mode  :character   Mode  :character  
##                                                                             
##                                                                             
##                                                                             
##      CAQF             AgeGroup          AgeGroupQF            Sex           
##  Length:391104      Length:391104      Length:391104      Length:391104     
##  Class :character   Class :character   Class :character   Class :character  
##  Mode  :character   Mode  :character   Mode  :character   Mode  :character  
##                                                                             
##                                                                             
##                                                                             
##     SexQF           InjuryLocation     InjuryLocationQF    InjuryType       
##  Length:391104      Length:391104      Length:391104      Length:391104     
##  Class :character   Class :character   Class :character   Class :character  
##  Mode  :character   Mode  :character   Mode  :character   Mode  :character  
##                                                                             
##                                                                             
##                                                                             
##  InjuryTypeQF       NumberOfAdmissions
##  Length:391104      Min.   :    0.00  
##  Class :character   1st Qu.:    0.00  
##  Mode  :character   Median :    1.00  
##                     Mean   :   46.52  
##                     3rd Qu.:   10.00  
##                     Max.   :61279.00
\end{verbatim}

\begin{Shaded}
\begin{Highlighting}[]
\CommentTok{\#NumberOFAdmissions Max. 61279}
\CommentTok{\#Column 1, Financial Year is a character string column not a integer: requires mutation. }
\CommentTok{\#"HBR" column same as orig\_hb co dataset column "HB"}


\FunctionTok{summary}\NormalTok{(orig\_uideaths)}
\end{Highlighting}
\end{Shaded}

\begin{verbatim}
##       Year          HBR               HBRQF                CA           
##  Min.   :2011   Length:182502      Length:182502      Length:182502     
##  1st Qu.:2013   Class :character   Class :character   Class :character  
##  Median :2016   Mode  :character   Mode  :character   Mode  :character  
##  Mean   :2016                                                           
##  3rd Qu.:2018                                                           
##  Max.   :2020                                                           
##      CAQF             AgeGroup          AgeGroupQF            Sex           
##  Length:182502      Length:182502      Length:182502      Length:182502     
##  Class :character   Class :character   Class :character   Class :character  
##  Mode  :character   Mode  :character   Mode  :character   Mode  :character  
##                                                                             
##                                                                             
##                                                                             
##     SexQF           InjuryLocation     InjuryLocationQF    InjuryType       
##  Length:182502      Length:182502      Length:182502      Length:182502     
##  Class :character   Class :character   Class :character   Class :character  
##  Mode  :character   Mode  :character   Mode  :character   Mode  :character  
##                                                                             
##                                                                             
##                                                                             
##  InjuryTypeQF       NumberofDeaths    
##  Length:182502      Min.   :   0.000  
##  Class :character   1st Qu.:   0.000  
##  Mode  :character   Median :   0.000  
##                     Mean   :   3.709  
##                     3rd Qu.:   1.000  
##                     Max.   :2759.000
\end{verbatim}

\begin{Shaded}
\begin{Highlighting}[]
\CommentTok{\#NumberofDeaths Max. 2759}

\CommentTok{\#Initial observations of datasets show a few discrepancies}
\end{Highlighting}
\end{Shaded}

\begin{Shaded}
\begin{Highlighting}[]
\NormalTok{cleaned\_admission }\OtherTok{\textless{}{-}}\NormalTok{ orig\_uiadmissions }\SpecialCharTok{\%\textgreater{}\%} 
  \FunctionTok{select}\NormalTok{( FinancialYear,}
\NormalTok{        HBR,}
\NormalTok{        CA,}
\NormalTok{        AgeGroup,}
\NormalTok{        Sex,}
\NormalTok{        InjuryType,}
\NormalTok{        InjuryLocation,}
\NormalTok{        NumberOfAdmissions) }\SpecialCharTok{\%\textgreater{}\%}
  \FunctionTok{clean\_names}\NormalTok{()  }\SpecialCharTok{\%\textgreater{}\%} 
  \FunctionTok{separate}\NormalTok{(financial\_year, }\AttributeTok{into =} \FunctionTok{c}\NormalTok{(}\StringTok{"year"}\NormalTok{), }\AttributeTok{sep =} \StringTok{"/"}\NormalTok{) }\SpecialCharTok{\%\textgreater{}\%} 
  \FunctionTok{filter}\NormalTok{(injury\_type }\SpecialCharTok{!=} \StringTok{"All Diagnoses"}\NormalTok{) }\SpecialCharTok{\%\textgreater{}\%} 
  \FunctionTok{filter}\NormalTok{(injury\_location }\SpecialCharTok{!=} \StringTok{"All"}\NormalTok{) }\SpecialCharTok{\%\textgreater{}\%} 
  \FunctionTok{filter}\NormalTok{(age\_group }\SpecialCharTok{!=} \StringTok{"All"}\NormalTok{) }\SpecialCharTok{\%\textgreater{}\%} 
  \FunctionTok{filter}\NormalTok{(sex }\SpecialCharTok{!=} \StringTok{"All"}\NormalTok{) }\SpecialCharTok{\%\textgreater{}\%} 
  \FunctionTok{mutate\_at}\NormalTok{(}\FunctionTok{c}\NormalTok{(}\DecValTok{1}\NormalTok{), as.numeric)}
\end{Highlighting}
\end{Shaded}

\begin{verbatim}
## Warning: Expected 1 pieces. Additional pieces discarded in 391104 rows [1, 2, 3,
## 4, 5, 6, 7, 8, 9, 10, 11, 12, 13, 14, 15, 16, 17, 18, 19, 20, ...].
\end{verbatim}

\begin{Shaded}
\begin{Highlighting}[]
\CommentTok{\#orig\_uiadmissions file assigned to new object after cleaning}
\CommentTok{\#select() function used to organise and select relevant columns from dataset: info, demographic, Injury Info, No. Admissions}
\CommentTok{\#clean\_names() function used to make snake case, lower case names for uniformity}
\CommentTok{\#separate() function to separate 2011/2012 and other values present in FinancialYear column into individual values}
\CommentTok{\#filter() function used to omit (!=) information or highlight (==)}
\CommentTok{\#mutate\_at(c(1), as.numeric) to change character values to numeric ones as column represent "Financial Year"}
     

\NormalTok{cleaned\_deaths }\OtherTok{\textless{}{-}}\NormalTok{ orig\_uideaths }\SpecialCharTok{\%\textgreater{}\%} 
  \FunctionTok{select}\NormalTok{( Year,}
\NormalTok{       HBR,}
\NormalTok{       CA,}
\NormalTok{       AgeGroup,}
\NormalTok{       Sex,}
\NormalTok{       InjuryType,}
\NormalTok{       InjuryLocation,}
\NormalTok{       NumberofDeaths) }\SpecialCharTok{\%\textgreater{}\%} 
  \FunctionTok{clean\_names}\NormalTok{() }\SpecialCharTok{\%\textgreater{}\%} 
  \FunctionTok{filter}\NormalTok{(injury\_type }\SpecialCharTok{!=} \StringTok{"All"}\NormalTok{) }\SpecialCharTok{\%\textgreater{}\%} 
  \FunctionTok{filter}\NormalTok{(injury\_location }\SpecialCharTok{!=} \StringTok{"All"}\NormalTok{) }\SpecialCharTok{\%\textgreater{}\%} 
  \FunctionTok{filter}\NormalTok{(age\_group }\SpecialCharTok{!=} \StringTok{"All"}\NormalTok{) }\SpecialCharTok{\%\textgreater{}\%} 
  \FunctionTok{filter}\NormalTok{(sex }\SpecialCharTok{!=} \StringTok{"All"}\NormalTok{)}

\CommentTok{\#Process from orig\_uiadmissions repeated}

\NormalTok{cleaned\_hb }\OtherTok{\textless{}{-}}\NormalTok{ orig\_hb }\SpecialCharTok{\%\textgreater{}\%} 
  \FunctionTok{select}\NormalTok{( HB,}
\NormalTok{       HBName,}
\NormalTok{       Country) }\SpecialCharTok{\%\textgreater{}\%} 
  \FunctionTok{clean\_names}\NormalTok{()}
  
\CommentTok{\#Process from orig\_uiadmissions repeated}
\end{Highlighting}
\end{Shaded}

\begin{Shaded}
\begin{Highlighting}[]
\NormalTok{full\_ui }\OtherTok{\textless{}{-}}\NormalTok{ cleaned\_admission }\SpecialCharTok{\%\textgreater{}\%}
  \FunctionTok{full\_join}\NormalTok{(cleaned\_deaths)}
\end{Highlighting}
\end{Shaded}

\begin{verbatim}
## Joining, by = c("year", "hbr", "ca", "age_group", "sex", "injury_type",
## "injury_location")
\end{verbatim}

\begin{Shaded}
\begin{Highlighting}[]
\CommentTok{\#full\_join() used to merge cleaned datasets together. full\_join() keeps values from both, easier to observe inconsistencies.}

\NormalTok{full\_ui\_hb }\OtherTok{\textless{}{-}}\NormalTok{ full\_ui }\SpecialCharTok{\%\textgreater{}\%} 
  \FunctionTok{full\_join}\NormalTok{(cleaned\_hb, }\AttributeTok{by =} \FunctionTok{c}\NormalTok{(}\StringTok{"hbr"} \OtherTok{=} \StringTok{"hb"}\NormalTok{))}

\CommentTok{\#full\_join() of final dataset. All datasets are now merged.}

\CommentTok{\# glimpse(full\_ui\_hb)}

\CommentTok{\# is.na(full\_ui\_hb)}


\NormalTok{clean\_full\_ui }\OtherTok{\textless{}{-}}\NormalTok{ full\_ui\_hb }\SpecialCharTok{\%\textgreater{}\%} 
  \FunctionTok{select}\NormalTok{(year,}
\NormalTok{      hbr,}
\NormalTok{      hb\_name,}
\NormalTok{      age\_group,}
\NormalTok{      sex,}
\NormalTok{      injury\_type,}
\NormalTok{      injury\_location,}
\NormalTok{      number\_of\_admissions,}
\NormalTok{      numberof\_deaths)  }\SpecialCharTok{\%\textgreater{}\%} 
  \FunctionTok{mutate}\NormalTok{(}\AttributeTok{numberof\_deaths =} \FunctionTok{coalesce}\NormalTok{(numberof\_deaths, }\DecValTok{0}\NormalTok{)) }\SpecialCharTok{\%\textgreater{}\%} 
  \FunctionTok{mutate}\NormalTok{(}\FunctionTok{across}\NormalTok{(}\FunctionTok{c}\NormalTok{(injury\_location), na\_if, }\StringTok{"Not Applicable"}\NormalTok{)) }\SpecialCharTok{\%\textgreater{}\%} 
  \FunctionTok{mutate}\NormalTok{(}\FunctionTok{across}\NormalTok{(}\FunctionTok{c}\NormalTok{(number\_of\_admissions), na\_if, }\DecValTok{0}\NormalTok{)) }\SpecialCharTok{\%\textgreater{}\%} 
  \FunctionTok{na.omit}\NormalTok{(clean\_full\_ui) }\SpecialCharTok{\%\textgreater{}\%} 
  \FunctionTok{group\_by}\NormalTok{(sex, age\_group, injury\_type, injury\_location, year, hbr, hb\_name) }\SpecialCharTok{\%\textgreater{}\%} 
  \FunctionTok{summarise}\NormalTok{(}\FunctionTok{across}\NormalTok{(}\FunctionTok{c}\NormalTok{(number\_of\_admissions, numberof\_deaths), sum)) }\SpecialCharTok{\%\textgreater{}\%} 
  \FunctionTok{arrange}\NormalTok{(age\_group) }\SpecialCharTok{\%\textgreater{}\%} 
  \FunctionTok{distinct}\NormalTok{() }\SpecialCharTok{\%\textgreater{}\%} 
  \FunctionTok{rowid\_to\_column}\NormalTok{()}
\end{Highlighting}
\end{Shaded}

\begin{verbatim}
## `summarise()` has grouped output by 'sex', 'age_group', 'injury_type',
## 'injury_location', 'year', 'hbr'. You can override using the `.groups`
## argument.
\end{verbatim}

\begin{Shaded}
\begin{Highlighting}[]
\CommentTok{\#Tidying and wrangling process to make data set ready for plotting.}

\CommentTok{\#Final modifications and wrangling of data.}
\CommentTok{\#Selected relevant columns for data plotting.}
\CommentTok{\#Mutate() and Coalesce() to change NA values in numberof\_deaths column to numerical value 0.}
\CommentTok{\#Mutate(), across() to na\_if functions to change character value "Not Applicable" to null value in column 7.}
\CommentTok{\#na.omit() to omit null, incomplete data present in merged data.}
\CommentTok{\#rowid\_to\_column() to number each line of the dataset for navigation and ease of use by adding integer column.}
\CommentTok{\#group\_by() to organise the columns into a desired order}

\CommentTok{\#This dataset is cleaned and ready to be observed for any last discrepancies before being plotted. }
\end{Highlighting}
\end{Shaded}

\begin{Shaded}
\begin{Highlighting}[]
\FunctionTok{glimpse}\NormalTok{(clean\_full\_ui)}

\NormalTok{clean\_full\_ui }\SpecialCharTok{\%\textgreater{}\%} 
  \FunctionTok{ungroup}\NormalTok{() }\SpecialCharTok{\%\textgreater{}\%} 
  \FunctionTok{count}\NormalTok{(year)}

\NormalTok{clean\_full\_ui}\SpecialCharTok{$}\NormalTok{year }\SpecialCharTok{\%\textgreater{}\%} 
  \FunctionTok{is.numeric}\NormalTok{()}

\CommentTok{\# Observations: range = 2011 to 2020, numeric values range = 2600 to 2900}

\NormalTok{clean\_full\_ui }\SpecialCharTok{\%\textgreater{}\%}
  \FunctionTok{ungroup}\NormalTok{() }\SpecialCharTok{\%\textgreater{}\%} 
  \FunctionTok{count}\NormalTok{(sex)}

\CommentTok{\# Female    13247, Male 14933   }

\NormalTok{clean\_full\_ui }\SpecialCharTok{\%\textgreater{}\%}
  \FunctionTok{ungroup}\NormalTok{() }\SpecialCharTok{\%\textgreater{}\%} 
  \FunctionTok{count}\NormalTok{(age\_group)}

\CommentTok{\# \#0{-}4 years    3570            }
\CommentTok{\# 5{-}9 years 3008        }
\CommentTok{\# 10{-}14 years   2793            }
\CommentTok{\# 15{-}24 years   3551            }
\CommentTok{\# 25{-}44 years   4123            }
\CommentTok{\# 45{-}64 years   4217            }
\CommentTok{\# 65{-}74 years   3337            }
\CommentTok{\# 75plus years  3581    }

\NormalTok{clean\_full\_ui }\SpecialCharTok{\%\textgreater{}\%}
  \FunctionTok{ungroup}\NormalTok{() }\SpecialCharTok{\%\textgreater{}\%} 
  \FunctionTok{count}\NormalTok{(injury\_type)}

\CommentTok{\# \#Accidental Exposure  3684            }
\CommentTok{\# Crushing  3120            }
\CommentTok{\# Falls 6246            }
\CommentTok{\# Other 5236            }
\CommentTok{\# Poisoning 3650            }
\CommentTok{\# Scalds    1832            }
\CommentTok{\# Struck by, against    4412    }

\NormalTok{clean\_full\_ui }\SpecialCharTok{\%\textgreater{}\%}
  \FunctionTok{ungroup}\NormalTok{() }\SpecialCharTok{\%\textgreater{}\%} 
  \FunctionTok{count}\NormalTok{(injury\_location)}

\CommentTok{\# Home  9903            }
\CommentTok{\# Other 7668            }
\CommentTok{\# Undisclosed   10609   }

\NormalTok{clean\_full\_ui }\SpecialCharTok{\%\textgreater{}\%}
  \FunctionTok{ungroup}\NormalTok{() }\SpecialCharTok{\%\textgreater{}\%} 
  \FunctionTok{count}\NormalTok{(hb\_name)}

\CommentTok{\#14 rows, }
\CommentTok{\# NHS Ayrshire and Arran    2393            }
\CommentTok{\# NHS Borders   1555            }
\CommentTok{\# NHS Dumfries and Galloway 1751            }
\CommentTok{\# NHS Fife  2331            }
\CommentTok{\# NHS Forth Valley  2102            }
\CommentTok{\# NHS Grampian  2644            }
\CommentTok{\# NHS Greater Glasgow and Clyde 2912            }
\CommentTok{\# NHS Highland  2272            }
\CommentTok{\# NHS Lanarkshire   2542            }
\CommentTok{\# NHS Lothian   2799    }
\CommentTok{\# NHS Orkney    695         }
\CommentTok{\# NHS Shetland  773         }
\CommentTok{\# NHS Tayside   2514            }
\CommentTok{\# NHS Western Isles 897 }

\NormalTok{clean\_full\_ui }\SpecialCharTok{\%\textgreater{}\%} 
  \FunctionTok{ungroup}\NormalTok{() }\SpecialCharTok{\%\textgreater{}\%} 
  \FunctionTok{count}\NormalTok{(number\_of\_admissions, sex)}

\NormalTok{clean\_full\_ui}\SpecialCharTok{$}\NormalTok{number\_of\_admissions }\SpecialCharTok{\%\textgreater{}\%} 
  \FunctionTok{is.numeric}\NormalTok{()}

\NormalTok{clean\_full\_ui }\SpecialCharTok{\%\textgreater{}\%} 
  \FunctionTok{ungroup}\NormalTok{() }\SpecialCharTok{\%\textgreater{}\%} 
  \FunctionTok{count}\NormalTok{(numberof\_deaths, sex)}

\NormalTok{clean\_full\_ui}\SpecialCharTok{$}\NormalTok{numberof\_deaths }\SpecialCharTok{\%\textgreater{}\%} 
  \FunctionTok{is.numeric}\NormalTok{()}

\CommentTok{\# unique(clean\_full\_ui)}

\FunctionTok{summary}\NormalTok{(clean\_full\_ui)}


\CommentTok{\#Final inspection of dataset before plotting. count() is used to check and determine total variables present in each column. ungroup() used to produce tibble we only specified column. unique() used to check for unique variables present in code. Code is printed but not run.}
\end{Highlighting}
\end{Shaded}

\begin{Shaded}
\begin{Highlighting}[]
\NormalTok{clean\_full\_ui}\SpecialCharTok{$}\NormalTok{age\_group }\OtherTok{\textless{}{-}} \FunctionTok{factor}\NormalTok{(clean\_full\_ui}\SpecialCharTok{$}\NormalTok{age\_group, }\AttributeTok{levels=}\FunctionTok{c}\NormalTok{(}\StringTok{"0{-}4 years"}\NormalTok{, }\StringTok{"5{-}9 years"}\NormalTok{, }\StringTok{"10{-}14 years"}\NormalTok{, }\StringTok{"15{-}24 years"}\NormalTok{, }\StringTok{"25{-}44 years"}\NormalTok{, }\StringTok{"45{-}64 years"}\NormalTok{, }\StringTok{"65{-}74 years"}\NormalTok{, }\StringTok{"75plus years"}\NormalTok{))}

\NormalTok{clean\_full\_ui\_filtered }\OtherTok{\textless{}{-}}\NormalTok{ clean\_full\_ui }\SpecialCharTok{\%\textgreater{}\%} 
  \FunctionTok{filter}\NormalTok{(injury\_type }\SpecialCharTok{==} \StringTok{"Falls"}\NormalTok{)}

\CommentTok{\#Final reorganizing of data as some discrepancies were observed in previous observations, including incorrect ordering of age groups, this has been fixed by manually assign the order of the age groups by the factor() function. }

\CommentTok{\#The final filtered data set is made after the penultimate data observations in the first graph have shown that "Falls" have the highest rate of admission. This filtered data set will be used to determine the rate of death in the final plot of this report.}
\end{Highlighting}
\end{Shaded}

\#\#Results:

\begin{Shaded}
\begin{Highlighting}[]
\CommentTok{\#The final, cleaned and organised first dataset is piped through the ggplot function, with appropriate x and y axis parameters set to display correlations }

\NormalTok{clean\_full\_ui }\SpecialCharTok{\%\textgreater{}\%} 
  \FunctionTok{ggplot}\NormalTok{(}\FunctionTok{aes}\NormalTok{(}\AttributeTok{x=}\NormalTok{injury\_type, }\AttributeTok{y=}\NormalTok{number\_of\_admissions,  }\AttributeTok{fill=}\NormalTok{age\_group)) }\SpecialCharTok{+} 
  \FunctionTok{geom\_bar}\NormalTok{(}\AttributeTok{position =} \StringTok{"dodge"}\NormalTok{, }
        \AttributeTok{stat =} \StringTok{"identity"}\NormalTok{,}
        \AttributeTok{width =} \FloatTok{0.8}\NormalTok{) }\SpecialCharTok{+}
  \FunctionTok{scale\_fill\_brewer}\NormalTok{(}\AttributeTok{palette =} \StringTok{"Pastel1"}\NormalTok{) }\SpecialCharTok{+}
  \FunctionTok{theme\_dark}\NormalTok{() }\SpecialCharTok{+}
  \FunctionTok{ggtitle}\NormalTok{(}\StringTok{"Number of Admissions by Injury Type"}\NormalTok{) }\SpecialCharTok{+}
  \FunctionTok{xlab}\NormalTok{(}\StringTok{"Injury Type"}\NormalTok{) }\SpecialCharTok{+}
  \FunctionTok{scale\_y\_log10}\NormalTok{(}\AttributeTok{name =} \StringTok{"Number Of Admissions (log10)"}\NormalTok{, }
          \AttributeTok{limits =} \FunctionTok{c}\NormalTok{(),  }
          \AttributeTok{breaks =} \FunctionTok{c}\NormalTok{(}\DecValTok{1}\NormalTok{, }\DecValTok{10}\NormalTok{, }\DecValTok{30}\NormalTok{, }\DecValTok{100}\NormalTok{, }\DecValTok{300}\NormalTok{, }\DecValTok{1000}\NormalTok{, }\DecValTok{10000}\NormalTok{)) }\SpecialCharTok{+}
  \FunctionTok{facet\_wrap}\NormalTok{(}\SpecialCharTok{\textasciitilde{}}\NormalTok{sex, }\AttributeTok{ncol =}\DecValTok{10}\NormalTok{) }\SpecialCharTok{+}
  \FunctionTok{coord\_flip}\NormalTok{() }\SpecialCharTok{+}
  \FunctionTok{guides}\NormalTok{(}\AttributeTok{fill=}\FunctionTok{guide\_legend}\NormalTok{(}\AttributeTok{title =} \StringTok{"Age Group"}\NormalTok{)) }
\end{Highlighting}
\end{Shaded}

\includegraphics{Health-Data-Programming-Assignment_files/figure-latex/data plot-1.pdf}

It can be observed from this grouped bar chart, that the highest rate of
admission over 2011 to 2020 was from females, aged 75+ years who had
suffered from a falling injury. Followed by males and females of 45-65
years who had been admitted due to suffering from a falling injury. The
X axis ``Number of Admissions'' was plotted using a logarithmic scale
due to the extremely high volume of admissions recorded for falling
injuries - as can be observed most other injuries, except ``Accidental
Exposure'' and ``Falls'' do not scale above 300 in data concerning
Number Of Admissions.

\begin{Shaded}
\begin{Highlighting}[]
\NormalTok{clean\_full\_ui\_filtered }\SpecialCharTok{\%\textgreater{}\%} 
  \FunctionTok{ggplot}\NormalTok{(}\FunctionTok{aes}\NormalTok{(}\AttributeTok{x=}\NormalTok{numberof\_deaths, }\AttributeTok{y=}\NormalTok{number\_of\_admissions, }\AttributeTok{color=}\NormalTok{age\_group)) }\SpecialCharTok{+} 
  \FunctionTok{geom\_point}\NormalTok{(}\AttributeTok{alpha =} \DecValTok{1}\NormalTok{, }\AttributeTok{size =} \FloatTok{1.7}\NormalTok{) }\SpecialCharTok{+}
  \FunctionTok{scale\_color\_brewer}\NormalTok{(}\AttributeTok{palette =} \StringTok{"Paired"}\NormalTok{) }\SpecialCharTok{+}
  \FunctionTok{theme\_dark}\NormalTok{() }\SpecialCharTok{+}
  \FunctionTok{ggtitle}\NormalTok{(}\StringTok{"Rate of Deaths in Falling Injury Admissions"}\NormalTok{) }\SpecialCharTok{+}
  \FunctionTok{xlab}\NormalTok{(}\StringTok{"Number of Deaths"}\NormalTok{) }\SpecialCharTok{+}
  \FunctionTok{scale\_x\_continuous}\NormalTok{(}\AttributeTok{limits =} \FunctionTok{c}\NormalTok{(),}
          \AttributeTok{breaks =} \FunctionTok{c}\NormalTok{(}\DecValTok{0}\NormalTok{, }\DecValTok{10}\NormalTok{, }\DecValTok{50}\NormalTok{, }\DecValTok{100}\NormalTok{, }\DecValTok{150}\NormalTok{, }\DecValTok{200}\NormalTok{, }\DecValTok{250}\NormalTok{, }\DecValTok{300}\NormalTok{),}
          \AttributeTok{minor\_breaks =} \FunctionTok{waiver}\NormalTok{()) }\SpecialCharTok{+}
  \FunctionTok{scale\_y\_log10}\NormalTok{(}\AttributeTok{name =} \StringTok{"Number Of Admissions (log10)"}\NormalTok{, }
          \AttributeTok{limits =} \FunctionTok{c}\NormalTok{(),  }
          \AttributeTok{breaks =} \FunctionTok{c}\NormalTok{(}\DecValTok{1}\NormalTok{, }\DecValTok{10}\NormalTok{, }\DecValTok{30}\NormalTok{, }\DecValTok{100}\NormalTok{, }\DecValTok{300}\NormalTok{, }\DecValTok{1000}\NormalTok{, }\DecValTok{10000}\NormalTok{)) }\SpecialCharTok{+}
  \FunctionTok{facet\_wrap}\NormalTok{(}\SpecialCharTok{\textasciitilde{}}\NormalTok{sex, }\AttributeTok{ncol =} \DecValTok{0}\NormalTok{) }\SpecialCharTok{+}
  \FunctionTok{coord\_flip}\NormalTok{() }\SpecialCharTok{+}
  \FunctionTok{guides}\NormalTok{(}\AttributeTok{color=}\FunctionTok{guide\_legend}\NormalTok{(}\AttributeTok{title =} \StringTok{"Age Group"}\NormalTok{)) }
\end{Highlighting}
\end{Shaded}

\begin{verbatim}
## Warning: `ncol` is missing or less than 1 and will be treated as NULL.
\end{verbatim}

\includegraphics{Health-Data-Programming-Assignment_files/figure-latex/unnamed-chunk-1-1.pdf}

The amount of death for falling injury admission can be observed as low
in comparison to quantity of admission. However, there is a positive
correlation between the recordings of admissions and death by falling
injury, in particular for females patients, aged 75+ years. This is an
understandable observation and confirms the observations made in the
previous plot. It can be therefore deduced that though the rate of
deaths is low by quantity comparison of admission data, a positive
correlation does exist proving that more deaths occur from falling
injury if there are higher rates of admissions.

\#Conclusion

To conclude, these data observations are limited to the NHS services who
have provided them and are subject to potential scrutiny regarding the
circumstances upon which the data was recorded, ie potential human
error. Most data processed has been compiled by omitting data that is
not complete or completely descriptive in its nature, this include ``Non
Applicable'' data or NA results. Furthermore, the NHS services specified
within the data set have registered at different times, which will mean
collected data is not uniform in its range - data cleaning and wrangling
has been performed as much as possible to compensate for this, resulting
in unusable data. This has not however interfered with the results
acquired through analyzing the data as the results show a clear spike in
recorded information, showing observable patterns in injuries and
demographics residing within Scotland.

\end{document}
